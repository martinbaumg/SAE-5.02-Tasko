\documentclass[12pt, a4paper]{article}
\usepackage[french]{babel}
\usepackage{caption}
\usepackage{graphicx}
\usepackage[T1]{fontenc}
\usepackage{listings}
\usepackage{geometry}
\usepackage{pgfplots}

\usepgfplotslibrary{polar}
\pgfplotsset{compat=1.12} 

\pgfplotsset{width=10cm,compat=1.9}
\usepackage[colorlinks=true,linkcolor=black,anchorcolor=black,citecolor=black,filecolor=black,menucolor=black,runcolor=black,urlcolor=black]{hyperref}
\usepackage{fancyhdr}
\pagestyle{fancy}
\lhead{}
\rhead{}
\chead{}
\rfoot{\thepage}
\lfoot{Martin Baumgaertner -- Mikhaïl Karapetyan -- Louis Pluviose -- Nicolas Wagner -- Thomas Strub -- Victor Uettwiller}
\cfoot{}

\renewcommand{\headrulewidth}{0.4pt}
\renewcommand{\footrulewidth}{0.4pt}

\begin{document}
\begin{titlepage}
	\newcommand{\HRule}{\rule{\linewidth}{0.5mm}} 
	\center 
	\textsc{\LARGE iut of colmar}\\[6.5cm] 
	\textsc{\Large SAE 5.02 piloter un projet informatique}\\[0.5cm] 
	\textsc{\large year 2023-24}\\[0.5cm]
	\HRule\\[0.75cm]
	{\Large\bfseries Tasko - Report on Our Project}\\[0.4cm]
	\HRule\\[0.5cm]
	\textsc{\large martin baumgaertner - mikhaïl karapetyan\\louis pluviose - nicolas wagner\\thomas strub - victor uettwiller}\\[6cm] 

	\vfill\vfill\vfill
	{\large\today} 
	\vfill
\end{titlepage}
\newpage
\tableofcontents
\newpage
\section{Introduction}
In today's fast-paced and collaborative work 
environments, effective task management is paramount 
to productivity and project success. "Tasko," a 
robust and user-friendly task management application, 
has been developed by a dedicated team of six talented 
individuals to meet the growing demands of modern task 
management. This report provides an in-depth analysis 
of Tasko, it's key features, and how it empowers users 
to effortlessly manage their tasks and projects. From 
task creation and assignment to online data backup and 
Docker deployment, Tasko offers a comprehensive 
solution for individuals and teams striving to enhance 
productivity and streamline collaboration. This 
report will delve into the core functionalities of 
Tasko, shedding light on its potential to revolutionize 
the way we organize, track, and accomplish our tasks.

\subsection{Features}
Tasko offers a comprehensive set of features designed 
to streamline task and subtask management while 
enhancing collaboration and productivity. Users 
can effortlessly create, organize, and track 
tasks and subtasks to simplify even the most 
complex projects. Task assignment capabilities 
promote effective teamwork and accountability, 
making it easy to delegate responsibilities. 
Online backup options, including Git repositories 
and web servers, ensure that your data remains 
secure and accessible, providing peace of mind. 
Deadlines and due dates can be set to efficiently 
manage tasks and meet project goals, while the 
prioritization feature enables users to focus on 
what matters most and stay on top of their workloads. 
Customizable labels allow for quick task categorization 
and easy identification. Furthermore, Docker deployment 
simplifies Tasko setup, providing a hassle-free and 
consistent environment for users.

\newpage
\section{Organization}
Swiftly, the need for effective organization
became evident as we embarked on the journey of
creating our project. It was at this point that I
stepped forward to take on the role of Product
Owner. My fellow team members graciously accepted
my offer, recognizing my diplomatic nature and
strong communication skills as valuable assets
for this role. Subsequently, Thomas expressed
his eagerness to assume the position of Scrum
Master. With these key roles defined, we proceeded
to divide the project into several distinct parts.
These segments encompassed the backend and database
development, frontend design, Docker/container
implementation, documentation, and the creation
of web HTML elements. Thomas, along with Mikhail
and Nicolas, took charge of the web HTML/frontend
portion, while Louis spearheaded the backend and
database aspects. Victor assumed responsibility
for the infrastructure related to Docker containers,
while I adopted a versatile role, lending assistance
wherever needed and primarily focusing on documentation.
The allocation of these roles was carefully considered
by the group members to foster harmonious collaboration
and to prevent disagreement.

\section{Difficulties Encountered}
One of the major challenges we encountered during our SAE
project was related to organizational issues. These issues
resulted in delays across various phases of the project.
The difficulties stemmed primarily from a cascading effect,
where one task's completion was dependent on another. 
For instance, the deployment phase couldn't proceed until
both the backend and frontend development were completed.
Similarly, the documentation process was also affected because
it relied on the completion of the software components.
This interdependency between project tasks had a significant
impact on our timeline. Delays in one area had a domino effect,
causing subsequent tasks to be pushed back as well. It became
evident that effective project management and coordination were
essential to mitigate these challenges.
The difficulties we encountered primarily revolved
around organizational issues and the interconnected nature of project
tasks. Addressing these challenges required a concerted effort to
streamline our workflow, prioritize tasks, and enhance communication
and coordination among team members. These experiences have provided
valuable insights into project planning and execution, which will be
beneficial for future endeavors.

\newpage
\section{Conclusion}
This project has been a valuable learning experience that has allowed
us to develop essential skills in organization, problem-solving, and
project management. One of the significant advantages of tackling a project
with a relatively broad scope was the opportunity to oversee every aspect
of the project from inception to completion.
Managing a project from start to finish has enabled us to gain a
comprehensive understanding of the challenges involved in software
development, from initial planning and design to coding, testing,
and deployment. It has also fostered a sense of autonomy and self-reliance,
encouraging us to think critically and solve problems as they arise.
However, it's worth noting that the size of our groups, consisting of
six members, presented its own set of challenges. While collaboration
is an essential aspect of teamwork, having a larger group often led
to inefficiencies in the distribution of tasks. Not all team members
could actively contribute to every aspect of the project due to its
size. In retrospect, smaller groups of three to four members might
have facilitated a more equitable distribution of responsibilities,
aligning better with the project's goal of involving everyone in all
project tasks.
Moreover, despite the challenges, this project has significantly
enriched our skills, knowledge, and experience in software development
and project management. It has prepared us to tackle more complex projects
in the future and has taught us the importance of effective teamwork,
communication, and adaptability in the face of challenges.


\end{document}